%%% Local Variables:
%%% mode: latex
%%% TeX-master: "<none>"
%%% End:

\title{Why to compress?}
\author{Vicente González Ruiz}

\maketitle
%\tableofcontents

\section{Why to compress?}
\begin{itemize}
\item Data compression can reduce the memory requirements of (almost) any kind of information source.
\end{itemize}

\section{Which data?}\label{which-data}
\begin{itemize}
\item Mainly \ldots{} audio, image, and video signals.
\end{itemize}

\section{Why these sources?}

\begin{itemize}
\item After the digitalization of any signal we get a sequence \(s[]\) of samples that represent the signal \(s\) with more or less fidelity.
\item Usually, \(s[]\) is encoded using PCM (Pulse Code Modulation), in which every sample \(s[i]\) is represented with the same number of bits.
\item Most digital PCM signals are memory demanding. For example, in a CD we have a data-rate of
\end{itemize}

\begin{equation}
  (16+16)\frac{\text{bits}}{\text{sample}}\times
  44{.}100\frac{\text{samples}}{\text{second}}=
  1{.}411{.}200\frac{\text{bits}}{\text{second}}.
\end{equation}

\begin{itemize}
\item Image and video signals require much more memory.
\end{itemize}

\section{Lab: audio comparison}
\href{https://nbviewer.jupyter.org/github/vicente-gonzalez-ruiz/why_to_compress/blob/master/audio_comparison.ipynb}{IPython notebook}

\section{Lab: image comparison)}
\href{https://nbviewer.jupyter.org/github/vicente-gonzalez-ruiz/why_to_compress/blob/master/image_comparison.ipynb}{IPython notebook}

\section{Memory requirements of PCM video}

\begin{itemize}
\item In RGB (PCM) video, each color pixel need at least 24 bpp
  (bits/pixel).
\item The memory requirements of RGB video are enormous. For example,
  an hour of \(640\times 480\times 25\) Hz true-color of PCM video
  needs:
\end{itemize}

\begin{equation}
  25\frac{\text{images}}{\text{second}}\times 640\cdot
  480\frac{\text{pixels}}{\text{image}}\times
  24\frac{\text{bits}}{\text{pixel}}=
  184{.}320{.}000\frac{\text{bits}}{\text{second}}
\end{equation}

\begin{equation}
  184{.}320{.}000\frac{\text{bits}}{\text{second}} \times
  3{.}600\frac{\text{seconds}}{\text{hour}} \times
  \frac{1~\text{G}}{1{.}024^3}\times
  \frac{1~\text{byte}}{8~\text{bits}} \approx 77~\text{Gbytes}
\end{equation}

\begin{itemize}
\item Video coding techniques should be used to compress this data. Most of these techniques are bases on Block-based Motion Estimation.
\end{itemize}

